% This is the Duke University Statistical Science LaTeX thesis template.
% It has been adapted from the Reed College LaTeX thesis template. The
% adaptation was done by Mine Cetinkaya-Rundel (MCR). Some of the comments
% that are specific to Reed College have been removed.
%
% Most of the work on the original Reed College document class and template
% was done by Sam Noble (SN). Later comments etc. by Ben Salzberg (BTS).
% Additional restructuring and APA support by Jess Youngberg (JY).
%
% See https://www.reed.edu/cis/help/latex/ for help. There are a
% great bunch of help pages there, with notes on
% getting started, bibtex, etc. Go there and read it if you're not
% already familiar with LaTeX.
%
% Any line that starts with a percent symbol is a comment.
% They won't show up in the document, and are useful for notes
% to yourself and explaining commands.
% Commenting also removes a line from the document;
% very handy for troubleshooting problems. -BTS

%%
%% Preamble
%%
% \documentclass{<something>} must begin each LaTeX document
\documentclass[12pt,twoside]{dukestatscithesis}
% Packages are extensions to the basic LaTeX functions. Whatever you
% want to typeset, there is probably a package out there for it.
% Chemistry (chemtex), screenplays, you name it.
% Check out CTAN to see: http://www.ctan.org/
%%
\usepackage{graphicx,latexsym}
\usepackage{amsmath}
\usepackage{amssymb,amsthm}
\usepackage{longtable,booktabs,setspace}
\usepackage{chemarr} %% Useful for one reaction arrow, useless if you're not a chem major
\usepackage[hyphens]{url}
% Added by CII
\usepackage{hyperref}
\usepackage{lmodern}
\usepackage{float}
\floatplacement{figure}{H}
% End of CII addition
\usepackage{rotating}

% Next line commented out by CII
%%% \usepackage{natbib}
% Comment out the natbib line above and uncomment the following two lines to use the new
% biblatex-chicago style, for Chicago A. Also make some changes at the end where the
% bibliography is included.
%\usepackage{biblatex-chicago}
%\bibliography{thesis}


% Added by CII (Thanks, Hadley!)
% Use ref for internal links
\renewcommand{\hyperref}[2][???]{\autoref{#1}}
\def\chapterautorefname{Chapter}
\def\sectionautorefname{Section}
\def\subsectionautorefname{Subsection}
% End of CII addition

% Added by CII
\usepackage{caption}
\captionsetup{width=5in}
% End of CII addition

% \usepackage{times} % other fonts are available like times, bookman, charter, palatino


% To pass between YAML and LaTeX the dollar signs are added by CII
\title{Duke Basketball Passing Networks}
\author{Sonia Xu}
% The month and year that you submit your FINAL draft TO THE LIBRARY (May or December)
\date{April 2018}
\advisor{Dr.~Alexander Volfovsky}
\institution{Duke University}
\degree{Bachelor of Science in Statistical Science}
\committeememberone{Dr.~Katherine Heller}
\committeemembertwo{Dr.~Scott Schmidler}
\dus{Dus X. Name}
%If you have two advisors for some reason, you can use the following
% Uncommented out by CII
% End of CII addition

%%% Remember to use the correct department!
\department{Department of Statistical Science}

% Added by CII
%%% Copied from knitr
%% maxwidth is the original width if it's less than linewidth
%% otherwise use linewidth (to make sure the graphics do not exceed the margin)
\makeatletter
\def\maxwidth{ %
  \ifdim\Gin@nat@width>\linewidth
    \linewidth
  \else
    \Gin@nat@width
  \fi
}
\makeatother

\renewcommand{\contentsname}{Table of Contents}
% End of CII addition

\setlength{\parskip}{0pt}

% Added by CII

\providecommand{\tightlist}{%
  \setlength{\itemsep}{0pt}\setlength{\parskip}{0pt}}

\Acknowledgements{
I want to thank a few people.
}

\Dedication{
You can have a dedication here if you wish.
}

\Preface{
This is an example of a thesis setup to use the reed thesis document
class (for LaTeX) and the R bookdown package, in general.
}

\Abstract{
The preface pretty much says it all. \par

Basketball has traditionally been viewed as a team sport, requiring the
collaboration of the entire team to successfully bring a ball to the
basket. The scope of this work attempts to model basketball possessions
for each game to explain team characteristics that can lead to
successful plays. Duke Basketball data from the 2014-2016 seasons was
converted into passing networks, with each node representing a unique
player and each edge containing information about a pass. These passing
networks not only represent the temporal qualities of basketball, but
also capture how interactions among players can lead to positive
outcomes during a game. Network characteristics, such as measures of
centrality, cohesiveness and latent position of the players, were
considered as predictors for play outcomes. This approach suggests that
successful Duke basketball teams have collaborative team dynamics. These
results can be applied to most team settings--uniform collaboration can
lead to better results than teams dominated by one leader.
}

% End of CII addition
%%
%% End Preamble
%%
%

\usepackage{amsthm}
\newtheorem{theorem}{Theorem}[chapter]
\newtheorem{lemma}{Lemma}[chapter]
\theoremstyle{definition}
\newtheorem{definition}{Definition}[chapter]
\newtheorem{corollary}{Corollary}[chapter]
\newtheorem{proposition}{Proposition}[chapter]
\theoremstyle{definition}
\newtheorem{example}{Example}[chapter]
\theoremstyle{definition}
\newtheorem{exercise}{Exercise}[chapter]
\theoremstyle{remark}
\newtheorem*{remark}{Remark}
\newtheorem*{solution}{Solution}
\begin{document}

% Everything below added by CII
  \maketitle

\frontmatter % this stuff will be roman-numbered
\pagestyle{empty} % this removes page numbers from the frontmatter
  \begin{acknowledgements}
    I want to thank a few people.
  \end{acknowledgements}
  \begin{preface}
    This is an example of a thesis setup to use the reed thesis document
    class (for LaTeX) and the R bookdown package, in general.
  \end{preface}
  \hypersetup{linkcolor=black}
  \setcounter{tocdepth}{2}
  \tableofcontents

  \listoftables

  \listoffigures
  \begin{abstract}
    The preface pretty much says it all. \par
    
    Basketball has traditionally been viewed as a team sport, requiring the
    collaboration of the entire team to successfully bring a ball to the
    basket. The scope of this work attempts to model basketball possessions
    for each game to explain team characteristics that can lead to
    successful plays. Duke Basketball data from the 2014-2016 seasons was
    converted into passing networks, with each node representing a unique
    player and each edge containing information about a pass. These passing
    networks not only represent the temporal qualities of basketball, but
    also capture how interactions among players can lead to positive
    outcomes during a game. Network characteristics, such as measures of
    centrality, cohesiveness and latent position of the players, were
    considered as predictors for play outcomes. This approach suggests that
    successful Duke basketball teams have collaborative team dynamics. These
    results can be applied to most team settings--uniform collaboration can
    lead to better results than teams dominated by one leader.
  \end{abstract}
  \begin{dedication}
    You can have a dedication here if you wish.
  \end{dedication}
\mainmatter % here the regular arabic numbering starts
\pagestyle{fancyplain} % turns page numbering back on

\chapter{Introduction}\label{introduction}

In basketball, a boxscore provides the statistical summary of the game
via defensive, offensive, and overall success metrics. The National
Basketball Association's records show that the first boxscore was
produced by the Boston Celtics in the 1946-1947 season. Initial records
kept track of basic basketball statistics for each player through
measures like minutes played (MP), field goals made (FGM), and free
throws made (FTM). Seventy-one years later, these metrics are still
popular today. While the National Basketball Association has boosted its
number of metrics to better summarize the game to include metrics like
rebounds per game (RBG), player efficiency rating (PER), free throw
attempts (FTA), and 3 field goals made (3FGM), these metrics still
cannot capture the entirety of the game because they do not take into
account the opposing team's defense/offense, nor previous plays that
significantly influenced the flow of the game.

Basketball is not the only sport that has encountered this modeling
problem. Soccer, a sport similar to basketball in that it requires a
team-oriented approach and it dynamically changes from moment to moment,
has also experienced a similar need by academia and major soccer teams
to better utilize the data to more fully understand the game. One
popular metric that has yet to be uniformly adopted is evaluating a
player's passing capabilities and team-value. Although a consensus has
yet to be adopted for the best metric, scholars from academia and the
National Basketball Association have sought to capture the game of
basketball more robustly in a similar fashion--via passing networks.

\chapter{Literature Review}\label{rmd-basics}

Passing forms the backbone of all team contact sports. To advance a ball
to the goal successfully, players must work together to
dribble/kick/throw the ball to its destination. Each pass to another
player can be considered a connection. These connections can be grouped
together to form a network of passes. Previous works have captured these
passing networks in soccer and basketball both statically and
dynamically--this literature review will explore the different methods
used to understand the value of a player and team, and the best
practices for modeling network data.

``Flow Motifs in Soccer: What can passing behavior tell us?'' by Joris
Bekkers and Shaunak Dabadghao was released in the 2017 MIT Sloan Sports
Analytics Conference, and focused on the static passing networks of the
last 4 seasons of 6 big European leagues with 8219 matches, 3532 unique
players and 155 unique teams. Passing sequences were denoted as a
sequence of all players involved five seconds before an attempted score.
This paper created radar graphs that illustrated the most popular
passing sequences by player, and compared radar graphs to identify
similar players. Passing sequences within teams were also compared
between teams by clustering the different passing styles of the
different teams. Key players were determined by the frequency that they
were included in the passing sequences.

``Exploring Team Passing Networks and Player Movement Dynamics in Youth
Association Football (Soccer)'' by Bruno Goncalves, Diogo Coutinho, Sara
Santos, Carlos Lago-Penas, Sergio Jimenez, and Jamie Sampaio compared
the passing sequences of two games played by two groups that differ in
age range, which showed that regardless of age, network centrality was
distinctive in both groups, and affirmed the long-held belief that more
passes lead to better game outcomes. Similar to the first paper, key
players were the ones most frequently involved in the passing sequences.
This paper created weighted graphs of the passing sequences, which
better visualized the passing structure of the team, and made it easier
to identify important players.

``Basketball Teams as Strategic Networks'' by Jennifer H. Fewell, Dieter
Armbruster, John Ingraham, Alexander Petersen, and James S. Waters
provided measurements to assess team entropy. First recording the
complete 30 seconds of a possession as a passing sequence, they
discovered that recording the last three nodes (players) before a shot
attempt was a better way to record passing sequences to avoid noisy
passing data. Although they were able to recognize various aspects of
team dynamics through weighted graphs like the second paper, they did
not find a consistent predictor of positive game outcomes. This paper
also identified that in general, teams typically range between two
playing styles: always passing to the best player or having no distinct
patterns in passing. These patterns can be noted by distinct betweenness
scores and uniform betweenness scores, respectively. Weighted graphs
clearly illustrated the two different playing styles. Also, the paper
found that the positions most involved with successful shots were: 1. PG
2. SG 3. SF 4. PF 5. CN.

Joachim Gudmundsson and Michael Horton summarised a variety of methods
that utilize object tracking data to analyze team and player
performances in ``Spatio-Temporal Analysis of Team Sports -- A Survey.''
Their research survey spanned modeling passing networks via graph theory
to calculating rebound probability with spatial coordinates. In
particular, work conducted by Daniel Cervone, Alex D'Amour, Luke Bornn,
and Kirk Goldsberry attempted to capture the game wholelistically via a
new measure called Expected Possession Value (EPV) in the paper ``A
Multiresolution Stochastic Process Model for Predicting Basketball
Possession Outcomes.'' This new metric uses three models--a
Microtransition Model, Macrotransition Entrance Model, and a
Macrotransition Exit Model--to capture the spatial biases of each player
and the in-game effects of pressure, so that it can measure the
likelihood of a successful play (made shot) given the previous sequence
of events. To compare players against the league-average scores, they
also calculated Expected Possession Value -Adjusted as an application
for teams.

Peter Hoff explains in ``Bilinear Mixed Effects Models for Dyadic
Data''" the structure of the AMEN package by describing the different
components of the model, which reinforces AMEN's suitability to model
network data. A Monte Carlo Markov algorithm, the model encompasses
modelling linear, bilinear, and dyadic covariates with multivariate
normal distributions. A dataset of international relations in Asia was
used to demonstrate the robustness of this model in revealing the
transitivity and clusterability of the observation.

Bailey Fosdick and Peter Hoff use AddHealth data in ``Testing and
Modeling Dependencies Between a Network and Nodal Attributes'' to
introduce a joint model that accounts for network factors and
attributes. The AddHealth dataset captures samesex friendship between
high school students, where students were asked to rank their top five
friends. Applying the model to this dataset via the AMEN package,
network features include rank information between students and nodal
attributes like exercise frequency of each student. Hoff and Fosdick
compare the performance of their joint model against a model that only
captures the effect of nodal attributes and show that the joint model
has a lower mean squared error in predicting missing values over a
20-fold cross validation. While the paper mainly focuses on
demonstrating the robustness of this model, there still exist challenges
in determining the level of dimensionality.

Peter Hoff in ``Modeling Homophily and Stochastic Equivalence in
Symmetric Relational Data'' proposes the benefits of modelling data in a
latent space. Models that transform datasets that contain network
features into latent space can capture two characteristics: homophily
and stochastic equivalence. Stochastic equivalence is when nodes can be
grouped based on similar characteristics, and homophily is when nodes
with similar characteristic nodes are more likely to have a relationship
than with different characteristic nodes. Models that measure these
relationships through latent eigenvalues perform better than models
measured through latent distance or latent class. This result constructs
the impetus for the AMEN package to utilize a latent eigenvalue model to
capture network and attribute data.

\chapter{Dataset}\label{dataset}

The dataset is from the Duke University Men's Basketball SportsVu
tracking data. Features were created by taking snapshots of the game
every 1/25th of a second and recording the player's location, action,
team, etc. Data was collected for each season from 2013-2016; the
dataset totals about 132,000 observations and 98 features. Since the
data is owned by the Duke Men's Basketball team, the data is private and
cannot be shared.

The dataset was presented in 3 different XML files:

Boxscore Data: This dataset shows the overall player statistics
(assists, points, rebounds) for each game.The Boxscore dataset was used
as a reference for player performance when modeling the posterior means
of the latent eigenvalue model.

Play by Play: This dataset provides a moment summary (dribble, foul,
pass) at time t for each game. The Play by Play dataset was used to
divide the raw data into possessions, which was then converted into
individual passing networks.

Sequence Optical: This dataset provides the locational summary of each
player for each game. The Sequence Optical dataset was used to map the
passing order for each possession.

\chapter{Data Cleaning}\label{data-cleaning}

Initially provided as XML files, the datasets were converted into csv
files and then merged to create a final dataset with 132,000
observations and 98 features. As this project primarily focuses on
passing, the data was converted into network data for each game. Each
game consists of an array of matrices that represent the passing count
between players for each possession.

Below is an example of a 10x10 matrix for a possession. The rows
indicate the passer, and the column indicates the receiver.
\begin{table}[H]
\centering\rowcolors{2}{gray!6}{white}
\begin{tabular}{l|r|r|r|r|r|r|r|r|r|r}
\hiderowcolors
\hline
  & 100023 & 100283 & 839023 & 456782 & 222789 & 134783 & 111124 & 098783 & 352671 & 213416\\
\hline
\showrowcolors
100023 & 0 & 1 & 0 & 3 & 0 & 0 & 0 & 0 & 0 & 0\\
\hline
100283 & 0 & 0 & 0 & 0 & 0 & 0 & 0 & 0 & 0 & 0\\
\hline
839023 & 0 & 1 & 0 & 0 & 0 & 0 & 0 & 0 & 0 & 0\\
\hline
456782 & 0 & 0 & 0 & 0 & 0 & 0 & 0 & 0 & 0 & 0\\
\hline
222789 & 0 & 0 & 0 & 0 & 0 & 0 & 0 & 0 & 0 & 0\\
\hline
134783 & 0 & 0 & 0 & 0 & 0 & 0 & 0 & 0 & 0 & 0\\
\hline
111124 & 0 & 0 & 0 & 0 & 0 & 0 & 0 & 0 & 0 & 0\\
\hline
098783 & 0 & 0 & 0 & 0 & 0 & 0 & 0 & 0 & 0 & 0\\
\hline
352671 & 0 & 0 & 0 & 0 & 0 & 0 & 0 & 0 & 0 & 0\\
\hline
213416 & 0 & 0 & 0 & 0 & 0 & 0 & 0 & 0 & 0 & 0\\
\hline
\end{tabular}
\rowcolors{2}{white}{white}
\end{table}
\section{Changes in Shot Clock Time}\label{changes-in-shot-clock-time}

As college basketball is a consistently changing sport, the NCAA changed
the play rules for the 2013-2014 college basketball season. Instead of a
35 second shot clock, the NCAA established a 30 second shot clock. Since
this work does not have a temporal component, the rule change does not
affect the results of model building drastically. However, the extra
five seconds may have allowed players to pass the ball more frequently,
which would affect the passing matrices.

\chapter{Exploratory Data Analysis}\label{exploratory-data-analysis}

Initial analysis of the data focused on understanding the many features
available in the Duke Men's Basketball dataset. This exploratory data
analysis explores shot attempt patterns through the years, as well as
potential biases with shot location.

\section{Changes in Shot Attempt
Patterns}\label{changes-in-shot-attempt-patterns}

As one of the best basketball programs in the nation, Duke University
Men's Basketball draws in a number of highly desirable and NBA-ready
recruits each year. For this, most players stay for only a year before
signing and playing for the National Basketball Association. A popular
trend for many skilled basketball players, this transition to
professional basketball has been coined by players as being
``one-and-done.'' Duke had two players (Rodney Hood, Jabari Parker)
drafted in the 2014 draft, three players (Jahlil Okafor, Justise
Winslow, and Tyus Jones) drafted in the 2015 draft, and one player
(Brandon Ingram) drafted in the 2016 draft. With so many players playing
the minimum in college, this paper concentrates on the analysis of
players who played more than one season with the Duke Men's Basketball
team, and had significant minutes with their time at Duke. With these
requirements, it is difficult to find the perfect player for analysis
because players like Marshall Plumlee, only had significant playing time
his senior year because it took time to fully develop him as a
competitive player.

Player 103929, on the other hand, serves as an interesting example
because he had consistent minutes for the 2013-2015 seasons. Player
103929's shot attempts were thus divided into each year to understand
how his shooting style has changed during his time at Duke.
\begin{figure}
\centering
\includegraphics{img/qc2013.png}
\caption{}
\end{figure}
Looking at the Player 103929's shot attempts for his junior season, he
was fairly even with his shooting, missing most of his 3 point shots,
and hitting most of his 2 point shots in the paint. It appears as though
he prefers to shoot from the right wing slightly more than he shoots
from the left wing.
\begin{figure}
\centering
\includegraphics{img/qc2014.png}
\caption{}
\end{figure}
In 2014, however, it can be noted that Player 103929 has transitioned to
shots that are closer to the basket and minimized the amount of 3 point
shot attempts. He brought his shot attempts closer inwards, which aligns
with the trend that he is better at shooting when he is closer to the
basket. Compared to 2013, he attacks more along the nail, which could be
attributed to Player 103929's growing strength as an off the dribble
jump shooter.
\begin{figure}
\centering
\includegraphics{img/qc2015.png}
\caption{}
\end{figure}
In the 2015 season, Player 103929 moves further out from the basket,
attempting more 3s. His preference for shooting in the right wing is
more pronounced. A new trend apparent from the graph, however, shows
that Player 103929 shoots more corner 3s than the previous two years.
While his shot attempts in the paint have slightly changed from 2013,
Player 103929 definitely has a unique playing style that has overall
been consistent in that he avoids shooting in the extended elbows and
short corners.

\section{Biases in Shot Location}\label{biases-in-shot-location}

While looking at the shot chart of each player shows their shooting
preferences, putting their shot chart in the context of Cameron is
another important aspect to note when analyzing a player's shot
preferences. Cameron Indoor Stadium's student section, known as the
Cameron Crazies, has been ranked as one of the best student sections in
the country by Bleacher Report, For The Win, and FOX Sports (to name a
few). Furthermore, during the first half, a team's offense is on the
opponent's side and a team's defense is on their home side. Thus, by
acknowledging where a player shoots in context to the location of the
fans and Coach K may reveal some biases to their shot location. Are
players showboating for the Cameron Crazies or are they showboating for
Coach K? To assess this trend, multiple Duke players were screened to
note any possible trends in shooting habits.

Intuitively, a player's shot chart distributon should be an even
reflection of the other half of the court (ie. if half court was
inflected onto the other half court, the shot distributions should be
similar). From Player 103929's Shot Charts, this intuition is true; it
is clear that he prefers shooting from the left side on both
sides--indicating that there does not exist an obvious bias in his shot
location based on exterior factors. However, when looking at a Player
842298, his shot attempts are more prevalent on Duke's side of the
bench, and less present on its complementary side. Perhaps, Player
842298 is showboating for his teammates or Coach K, and plays off of the
exterior factors in a game. Further analysis will be conducted in later
iterations of this paper to better understand this bias.

\includegraphics{img/shotdistribution.png}

\chapter{Passing Networks}\label{passing-networks}

The main motivation behind this project is to understand the passing
structure of Duke players in a game to create a better metric to
evaluate players in the game of basketball. For this, each game was
decomposed into individual possessions. Players who are in possession of
the ball during each of the possessions are identified as vertices, and
their passes to other players are edges in a pass network. Each vertex
contains attributes about the player (e.g.~fouls in the game), and each
edge contains attributes about the pass (ie. distance passed).

\section{A Breakdown of a Passing
Network}\label{a-breakdown-of-a-passing-network}

\subsection{Game Network}\label{game-network}

Below is an example of a passing network for an entire game, where each
number represents the unique id of a player.
\begin{figure}
\centering
\includegraphics{img/gamenetwork_ex.png}
\caption{}
\end{figure}
\subsection{Possession Network}\label{possession-network}

Breaking it down into a single game possession, the network becomes
reduced to a smaller network. One challenge in identifying a posession
was the inconsistency of the dataset's shot clock. For this, a new
\emph{possession} for this paper is defined as the moment when a team
turns over the ball to the other team. For this, a possession may
contain more than five players if players sub in/out within a
possession.
\begin{figure}
\centering
\includegraphics{img/possessionnetwork_ex.png}
\caption{}
\end{figure}
\subsection{A Vertex and an Edge}\label{a-vertex-and-an-edge}

A single pass between player \(109412\) and \(109413\) has a thin line
because it only occurred once during this game. The arrow indicates the
direction of the pass, and when checking the edge attribute between
these two vertices, the distance of the pass between \(109412\) and
\(109413\) is 22.83 units. Looking at vertex attributes, player
\(109412\)'s position is a guard.
\begin{figure}
\centering
\includegraphics{img/passnetwork_ex.png}
\caption{}
\end{figure}
\section{Initial Analysis of Passing
Networks}\label{initial-analysis-of-passing-networks}

Simply looking at a graph can reveal important characteristics about a
player's role within a team. On a possession level, if a player receives
many passes (as noted by a thicker edge), then he has a more central
role on the team, and his teammates clearly rely on him to make good
passs.

Other interesting network calculations are betweeness centrality; this
metric can be visualized by the passing network, and noted as the
popularity of a player based on how connected/central he is to the play.
For this, returning to the Duke vs.~Davidson game, we can note that
Player \(109415\) is an important and valuable player for Duke because
his betweenness centrality score is the highest score as denoted by the
table below:
\begin{tabular}{r|r|r|r|r|r|r|r|r|r}
\hline
109412 & 109781 & 121033 & 116519 & 116520 & 109413 & 97330 & 97328 & 121034 & 109415\\
\hline
21.7 & 21.5 & 14.5 & 12.5 & 7.5 & 5.5 & 2 & 1.8 & 0.5 & 0\\
\hline
\end{tabular}
Furthermore, we can presume that players who are most connected to the
ball should be able to best handle the ball. For this, we expect the
players with the highest betweeness score to be the starters for Duke's
2013-2014 Men's Basketball team. Checking the starting line-up from Duke
Men's Basketball for the 2013-2014 season, the betweenness score
correctly matches Coach K's starting line-up.

\chapter{Network Modeling}\label{network-modeling}

\section{Posession Analysis}\label{posession-analysis}

Each possession in basketball typically ends in a made or missed shot,
turnover, offensive or defensive rebound, or foul. Each possession has
network characteristics unique to the play--number of triangles, passing
reciprocity, betweenness centrality, etc. Using possession-level network
characteristics to predict the outcome of a play can shed light on the
utility value of certain team characteristics. If high centrality is a
significant predictor of successful shots, then having a superstar
player is the better playing style for basketball.

\section{Multinomial Model}\label{multinomial-model}

A binomial logistic regression was initially fit to determine what
features were important to predicting a made shot. Predictors included
variables like number of triangles, passing reciprocity, betweenness
centrality,.. Below is the formula:

\[y_{made,i} \sim \beta_{tri}x_{tri,i} + \beta_{recip}x_{recip,i} + \beta_{betweenness}x_{betweenness,i}... + \epsilon_{i}\]
for posession \[i\]

A multinomial model was used to explain the data more robustly by
binning the outcome of possessions more broadly. Three categories were
created: good outcomes (made shot, offensive rebound), bad outcomes
(missed shot, turnover), and neutral outcomes (inbound ball).

\section{Results}\label{results}

Network characteristics, although informative in summarizing
possessions, were not significant predictors of the outcome of a
basketball possession. Higher reciprocity and triangle dependence,
indicators of collaborative teamwork, had a direct relationship with
made shots. These results from the preliminary model are largely
directional, reaffirming the notion that network characteristics are not
informative enough to capture the game of basketball.

\section{AMEN Analysis}\label{amen-analysis}

The AMEN package implements a latent eigenvalue model through a Monte
Carlo Markov Chain. Hoff explains his preference for utilizing a latent
eigenvalue model in ``Modeling Homophily and Stochastic Equivalence in
Symmetric Relational Data.'' This model captures the game of basketball
more robustly than a multinomial model because it doubly captures
network and nodal attributes. The output of this model provides
posterior means of the row and column effects, as well as higher order
dependence covariates for each player. Overall game performance for each
player can be used as a response against the output of the latent
eigenvalue model in order to check if a player's network and nodal
attributes are significant influencers; points per game for each player
was used as a response for a Poisson regression that used the posterior
means of the latent eigenvalue model as its features.

\section{Latent Eigenvalue Model \& Poisson
Regression}\label{latent-eigenvalue-model-poisson-regression}

\[y_{ij} =  \beta_{i}x_{i} + r_{i} + s_{i} + u^{T}v + \epsilon_{i}\]
where \[r_{i} = \beta_{i}x_{i} + a_{i}\] and
\[s_{i} = \beta_{i}x_{i} + b_{i}\]

Nodal Attributes: Points per game for each player

Nodal Features: Was in previous player (0/1), currently in possession
(0/1)

Dyadic Features: Shared position, shared height, shared weight, shared
class

Network Attributes: Passing network matrix

\section{Structural Zeros}\label{structural-zeros}

Currently, the AMEN package takes on a pxp matrix of players to model
the passing relationship between players. Currently, the model uses data
from the 2014-2015 season, so p = 10. These 10 players represent the
five players on the court, and the five players on the bench. For the
five players on the court, if there does not exist a pass between two
players, then a zero populates the matrix to account for the nonevent.
However, if a player is on the bench, he similarly cannot receive a
pass, so all bench players will always have a zero populated in the
10x10 matrix. This creates a challenge in modeling the data because the
zeros in the matrix represent two different events--players who had the
possibility to receive the ball but did not and players who never had
the chance to receive the ball.

The model includes a binary column feature that signals a player's
status (on court or on bench). While it does not solve the structural
zero problem entirely, this feature accounts for the differences between
active and nonactive players.

\section{Model Fit}\label{model-fit}

The model was fit at two stages: latent eigenvalue model and poisson
regression.

\subsection{AMEN Fit}\label{amen-fit}

Checking the model fit of the AMEN output, the posterior predictive
checks perform well for the column effects. A challenge with this model,
as noted at the end of Fosdick's and Hoff's ``Testing and Modeling
Dependencies Between a Network and Nodal Attributes,'' was determining
the appropriate dimensionality for transforming the data onto a latent
space. Initially, the data was fit with a dimensionality of 2. However,
as there are always five players on the court, higher dimensionality
would be more appropriate for this model. The data was fit again with a
dimensionality of four, and there was not a significant improvement in
the fit of the posterior predictive checks. This suggests that a
dimensionality of 2 is enough to capture the data.

The output below shows the model fit of a game with a dimensionality of
2. \includegraphics{img/amenoutput.png}

The output below shows the model fit of a game with a dimensionality of
4. \includegraphics{img/amenoutput4.png}

\subsection{Poisson Regression Fit}\label{poisson-regression-fit}

The output of the AMEN model was plugged into a Poisson regression, with
the response as a player's points per that game. Assessing the fit of
the Poisson model, it assumes hetereoscedascity, and fits the checks
well. However, the p-value for the deviance goodness of fit test was
\[9.152\] x \(10^{-25}\), which indicates with strong evidence that the
model fits the data poorly. Regardless, this model provides a
directional indication that row and column effects (passing and
receiving) are significant features in predicting the productivity of a
player.

Below is the summary output of the Poisson regression.
\includegraphics{img/poissonsummary2.png}

\section{Results}\label{results-1}

The latent eigenvalue model with less dyadic features (only shared
position) performed comparably to the full model based on the posterior
predictive checks. The output of the latent eigenvalue model was used to
predict a player's points per game via a Poisson Regression. While both
the row and column posterior means were significant influencers of a
player's points per game, row effects had a larger coefficient estimate.
These results confirm the importance of passing and teamwork for
successful plays.

Exploratory analysis of the model output reveals that there are
game-level differences in play style for certain players. Comparing the
performance of two games with two different outcomes for the Duke
2014-2015 team, there a noticeable differences in the sender and
receiver effects. Player 126160, for instance, had higher receiver
effects in a loss compared to a win. This result could imply that player
126160 was not fulfilling his role on the team if his performance
recorded by a successful game was his baseline. On the other hand, some
players did not drastically change betweeen a a win and a loss. Players
like Player 109415 who typically had low playing time and thus lower
sender and receiver effects overall, intuitively had even lower sender
and receiver effects during a loss. The loss of a team, although
influenced by many characteristics, can be partially attributed to these
differences in individual player performance. However, this model does
not take into account sufficient statistics for the varying defense the
Duke teams play.\\
\includegraphics{img/sendreceiver.png}

\chapter{Conclusion}\label{conclusion}

Duke Men's Basketball has a vast and rich dataset that has much to be
explored. Of particular interest is how a player interacts against his
teammates and defenders. This paper focuses on modeling player
interactions via passing networks--network centrality and betweenness
scores identify key players within a team. By evaluating passing
networks, not only can a player's value within a team be deduced, but
also how a player's value within a team has changed over time. Modeling
each posession in a game with network characteristics as features can be
directionally useful. A more robust approach utilizes Peter Hoff's AMEN
package, which models both nodal and network characteristics. The
results through this approach similarly show the significance of passing
and receiving the ball. Teamwork and high collaboration leads to
successful plays.

\section{Future Steps}\label{future-steps}

The scope of this work captures possessions of a game on an individual
level. However, using the output of the latent eigenvalue model to
predict nodal attributes is not sufficient to capture the game fully.
Currently, an implementation and adaptation of a model influenced by
Luke Bornn's ``A Multiresolution Stochastic Process Model for Predicting
Basketball'' aims to capture the game of basketball more robustly.
Future work will use this advanced model to create metrics for assessing
a player's production value on a team level. A summary of the model
replication can be found in Appendix A.

\chapter{Appendix A}\label{appendix-a}

\section{Model Replication}\label{model-replication}

The initial approach to understand how to best capture passing networks
sought to replicate Daniel Cervone, Alex, D'Amour, Luke Bornn, and Kirk
Goldsberry's paper,``A Multiresolution Stochastic Process Model for
Predicting Basketball Possession Outcomes.'' They attempt to capture the
game wholelistically via a new measure called Expected Possession Value
(EPV). This new metric uses three models--a Microtransition Model,
Macrotransition Entrance Model, and a Macrotransition Exit Model--to
capture the spatial biases of each player and the in-game effects of
pressure, so that it can measure the likelihood of a successful play
(made shot) given the previous sequence of events. To compare players
against the league-average scores, they also calculated Expected
Possession Value -Adjusted as an application for teams. Below is a brief
overview of each model.

This paper is particularly interesting because EPV utilizes the
spatio-temporal elements of the game, so it models the NBA game
dynamically. Given Duke Basketball data, the motivation is to replicate
``A Multiresolution Stochastic Process Model for Predicting Basketball
Possession Outcomes,'' to better understand the Duke Men's team, as well
as to compare professional basketball to collegiate basketball
individual and team playing styles. Below is a brief overview of each
model used in the paper to calculate EPV.

\subsection{Microtransition Model}\label{microtransition-model}

\(x^{l}(t+\epsilon) = x^{l}(t) + \alpha^{l}_{x}[x^{l}(t) - x^{l}(t-\epsilon)] + \eta^{l}_{x}(t)\)
where
\(\eta^{l}_{x}(t) \sim N(\mu^{l}_{x}(z^{l}(t)), (\sigma^{l}_{x})^{2})\)

The microtransition model models the defensive conditions of the game
based on the \((x,y)\) coordinates of a player and their acceleration
effects (\(\alpha^{l}_{x}(t)\)). It is also assumed that a player's
spatial location is normally distributed. Since players play
differently, each microtransition model is specifically fitted to the
player.

\subsection{Macrotransition Entrance
Model}\label{macrotransition-entrance-model}

\(P(M(t)|F_{t}^{(Z)}\) The macrotransition entrance model predicts
whether the next move will be a pass (4 options), shot attempt, or
turnover. The model is disjoint.

\subsection{Macrotransition Exit
Model}\label{macrotransition-exit-model}

\(P(C_{\delta_{t}}|M(t), F_{t}^{(Z)})\) Given the Macrotransition
Entrance Model predicts a shot attempt, it indexes to a logistic
regression model to calculate player \(l\)'s successful shot
probability. Given the Macrotransition Entrance Model predicts a pass,it
indexes to a model that predicts where the pass will take place.
Otherwise, a turnover is assumed.

\subsection{Fall Backs on the Implementation of this
Model}\label{fall-backs-on-the-implementation-of-this-model}

Currently, the implementation of the model has yet to be completed due
to setbacks of incompatible R code. The implementation of this paper is
currently still in progress.

\subsection{Proposal}\label{proposal}

Regardless, we hypothesize that since both metrics are calculated via a
semi-Markov process, EPV fails to capture the full nature of the
possession because it only uses the last posession as a prior. The model
would be more robust if it captured the entirety of the posession in its
prior--however, the computational time of such an ordeal would prevent
any real-time analyses. Thus, this paper proposes that a simpler model
may perform more quickly and potentially just as robustly to allow for
game-time analyses.

\backmatter

\chapter*{References}\label{references}
\addcontentsline{toc}{chapter}{References}

Bekkers, J., \& Dabadghao, S. (2017). ``Flow Motifs in Soccer: What can
passing behavior tell us? Sloan Sports Analytics Conference. Retrieved
from
\url{http://www.sloansportsconference.com/wp-content/uploads/2017/02/1563.pdf}

Cervone, D., D'Amour, A., Bornn, L., Goldsberry, K. (2016). ``A
Multiresolution Stochastic Process Model for Predicting Basketball
Possession outcomes.'' Retrieved from
\url{https://arxiv.org/pdf/1408.0777.pdf}

Fewell, J., Ambruster, D., Ingraham, J., Petersen, A., \& Waters, J.
(2012). ``Basketball Teams as Strategic Networks.'' PLOS. Retrieved from
\url{http://journals.plos.org/plosone/article?id=10.1371/journal.pone.0047445}

Fosdick, B.K., Hoff, P.D. (2013). ``Testing and Modeling Dependencies
Between a Network and Nodal Attributes.'' Retrieved from
\url{https://arxiv.org/abs/1306.4708}

Goncalves, B., Coutinho, D., Santos, S., Lago-Penas, C., Jimenez, S., \&
Sampaio, J. (2017). ``Exploring Team Passing Networks and Player
Movement Dynamics in Youth Association Football.'' PLOS. Retrieved from
\url{https://doi.org/10.1371/journal.pone.0171156}.

Gudmundsson, J., \& Horton, M. (2016). ``Spatio-Temporal Analysis of
Team Sports -- A Survey.'' Retrieved from
\url{https://arxiv.org/abs/1602.06994}

Hoff, P.D. (2007). ``Modeling Homophily and Stochastic Equivalence in
Symmetric Relational Data.'' Retrieved from
\url{https://arxiv.org/abs/0711.1146}.

Hoff, P.D. (2003). ``Bilinear Mixed Effects Models for Dyadic Data.''
Retrieved from
\url{https://www.stat.washington.edu/~pdhoff/Preprints/dyadic.pdf}.

\markboth{References}{References}

\noindent

\setlength{\parindent}{-0.20in} \setlength{\leftskip}{0.20in}
\setlength{\parskip}{8pt}

\hypertarget{refs}{}
\hypertarget{ref-angel2000}{}
Angel, E. (2000). \emph{Interactive computer graphics : A top-down
approach with opengl}. Boston, MA: Addison Wesley Longman.

\hypertarget{ref-angel2001}{}
Angel, E. (2001a). \emph{Batch-file computer graphics : A bottom-up
approach with quicktime}. Boston, MA: Wesley Addison Longman.

\hypertarget{ref-angel2002a}{}
Angel, E. (2001b). \emph{Test second book by angel}. Boston, MA: Wesley
Addison Longman.


% Index?

\end{document}
